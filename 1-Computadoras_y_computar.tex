% Created 2018-03-06 mar 18:44
\documentclass[11pt]{article}
\usepackage[utf8]{inputenc}
\usepackage[T1]{fontenc}
\usepackage{fixltx2e}
\usepackage{graphicx}
\usepackage{longtable}
\usepackage{float}
\usepackage{wrapfig}
\usepackage{rotating}
\usepackage[normalem]{ulem}
\usepackage{amsmath}
\usepackage{textcomp}
\usepackage{marvosym}
\usepackage{wasysym}
\usepackage{amssymb}
\usepackage{hyperref}
\tolerance=1000
\author{Ismael}
\date{\today}
\title{1-Computadoras\_y\_computar}
\hypersetup{
  pdfkeywords={},
  pdfsubject={},
  pdfcreator={Emacs 25.3.1 (Org mode 8.2.10)}}
\begin{document}

\maketitle
\tableofcontents

\section{Resumen}
\label{sec-1}
Una computadora en el sentido más general, es tan sólo un dispositivo que acepta datos o una entrada y la procesa de manera de obtener automáticamente un resultado. Cuando una computadora está haciendo algún tipo de trabajo, sea que esté abriendo una aplicación, editando una imagen o reproduciendo una canción, eatá \textbf{computando}. Computar, en el sentido más general, significa calcular. Para que una computadora pueda operar de correctamente, diferentes partes de la computadora deben comunicarse e interactuar entre ellas de la manera correcta.

\section{Entradas y salidas}
\label{sec-2}
Una computadora comienza tomando algún tipo de datos o información, llamada \textbf{entrada}. La entrada puede tomar una variedad de formas - clicks del ratón, teclas, toques en una pantalla táctil o botones. La entrada también puede tomar formas menos tradicionales: como la forma en que los detectores de humo toman información del ambiente, o la forma en que los autos toman la información del volante para determinar hacia dónde doblar.

Las computadoras usan la entrada que se les suministra para generar un resultado. En ciencias de la computación, este resultado es llamado \textbf{salida}. En el caso tradicional de una computadora de escritorio, la salida toma la forma de lo que sea que se muestre en la pantalla del usuario. Pero la salida puede tomar muchas otras formas, como producir un sonido o causar movimiento.

De alguna manera las computadoras necesitan traducir las entradas en salidas, procesando la información de la entrada de forma de generar la salida necesaria. Este proceso toma la forma de un \textbf{algoritmo}, el cual es un conjunto de órdenes que una computadora debe seguir para trducir las entradas en las salidas deseadas. \textbf{Programar} es el proceso de proveer a una computadora con un conjunto de instrucciones, o un algoritmo, de forma de realizar una tarea en particular.

\section{El proceso computacional}
\label{sec-3}
El proceso de traducir entradas en salidas es conocido como \textbf{proceso computacional}, e involucra realizar una serie de cálculos en forma de un algoritmo.
El proceso computacional puede variar en complejidad y en el número de pasoso requeridos para completar una tarea. A veces, el proceso computacional es relativamente simple, como el proceso de calcular 5+3. En otros casos, muchas tareas computacionales son mucho más complicadas. Todas las tareas que realiza una computadora, como calcular la ruta de casa al liceo con el GPS, o poner una alarma a determinada hora, requiere computación.

\section{Cómo funciona una computadora}
\label{sec-4}
Cada parte de una computadora sirve para una función específica, y juntas permiten a la computadora realizar tareas. Las computadoras requieren una combinación de \textbf{hardware}, las partes físicas que componen una computadora, y \textbf{software}, los programas e instrucciones que corren en la computadora. Gran parte del hardware está unido a la placa madre (motherboard), que contiene el hardware que ayuda a las diferentes partes a comunicarse entre ellas.

Las computadoras requieren electricidad para funcionar, por lo tanto deben tener una fuente de poder. Cuando el botón de encendido es presionado, la fuente de poder comienza a alimentar de energía a la computadora, lo cual comienza el proceso de encendido de la misma.

Luego de que el hardware de la computadora ha sido inicializado, el próximo paso es preparar el software, comenzando por el \textbf{sistema operativo}, el cual es el software que administra la ejecucuón de otros programas en la máquina (sistemas operativos comunes son Windows, Linux y macOS). El sistema operativo, como otros sistemas de archivos de la computadora está guardado en el disco duro de la computadora, el cual es el método principal de almacenamiento. Cada computadora tiene además una unidad central de proceso (\textbf{CPU}), referida como el "procesador", que es responsable de correr el software y ejecutar cómputos.
% Emacs 25.3.1 (Org mode 8.2.10)
\end{document}
